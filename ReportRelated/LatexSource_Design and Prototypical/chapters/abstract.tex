\chapter*{Abstract}
\addcontentsline{toc}{chapter}{Abstract}


\selectlanguage{english} 
The Internet of Things (IoT) offers many advantages to our society, including benefits regarding the economy and human convenience. While these are not empty promises, IoT devices have the major drawback of being inherently vulnerable to malware due to various characteristics. As the number of IoT devices is expected to triple by 2030, possible defense mechanisms against such malware (e.g. Bashlite or Mirai) are essential. This thesis proposed and implemented a prototype of a cooperative and reactive Moving Target Defense (MTD) framework that exploits the weaknesses of Bashlite, a well-known IoT malware. The first weakness is the ability to disrupt the connection of a Bashlite client from the Bashlite server by changing the client's IP address. The second vulnerability is that Bashlite scans and distributes itself via the Telnet port 23. Hence, the infected device is instructed to change its local IP address to disconnect itself from the Bashlite server, and the other devices in the network are instructed to temporarily move their Telnet service port to hide until Bashlite is rendered harmless. 

Three different evaluation scenarios were created, all consisting of two virtual machines, one of which is infected with Bashlite that attempts to infect the second machine. The scenarios differed in the inclusion of the cooperative component and the trigger of the execution of the MTD techniques. The two possibilities for the trigger were proactive (every minute) and reactive (after the detection of Bashlite). The evaluation scenarios have shown that the proposed cooperative and reactive framework and techniques have significant advantages over a non-cooperative and reactive approach and a cooperative but proactive approach. In addition to halving the overall infection time in the system, the overall availability of the machines, defined by outgoing packet losses and outgoing and incoming Telnet connections, was also significantly improved. In addition, the CPU and RAM usage of the framework and techniques executed were minimal. Although the cooperative and reactive approach provided by far the best results, each MTD approach has its advantages and further research is required to make use of this promising defense mechanism.

\clearpage
\selectlanguage{german}
Das Internet der Dinge (Internet of Things, IoT) bringt unserer Gesellschaft viele Vorteile, unter anderem im Bereich der Wirtschaft und des menschlichen Komforts. Das sind zwar keine leeren Versprechungen, allerdings haben IoT Geräte den grossen Nachteil, dass sie aufgrund verschiedener Eigenschaften stark anfällig für Schadsoftware sind. Da sich zusätzlich die Zahl der IoT-Geräte bis 2030 voraussichtlich verdreifachen wird, sind mögliche Abwehrmechanismen gegen solche Schadsoftware (z. B. Bashlite oder Mirai) von entscheidender Bedeutung. In dieser Thesis wurde ein Prototyp eines Frameworks für ein kooperatives und reaktives Verteidigungssystem (Moving Target Defense, MTD) vorgeschlagen und implementiert. Dieser Verteidigungsmechanismus nutzt die Schwachstellen von Bashlite, einer bekannten IoT Malware, aus. Die erste Schwachstelle ist die Möglichkeit, die Verbindung eines Bashlite-Clients mit dem Bashlite-Server zu unterbrechen, indem die IP-Adresse des Clients geändert wird. Die zweite Schwachstelle besteht darin, dass Bashlite den Telnet-Port 23 von anderen Maschinen scannt und sich darüber auch verbreitet. Wegen diesen zwei Schwachstellen wird die infizierte Maschine angewiesen, die lokale IP-Adresse zu ändern, um sich vom Bashlite-Server zu trennen, und die anderen Geräte im Netzwerk werden angewiesen, ihren Telnet-Service-Port vorübergehend auf einen anderen Port zu legen, bis Bashlite unschädlich gemacht ist.

Drei unterschiedliche Bewertungsszenarien wurden erstellt, wobei alle aus zwei virtuellen Maschinen bestehen, von denen eine mit Bashlite infiziert ist, die dann versucht die zweite Maschine zu infizieren. Die Szenarien unterschieden sich durch die Einbeziehung der kooperativen Komponente und durch den Auslöser für die Ausführung der MTD-Techniken. Die beiden Möglichkeiten für den Auslöser waren proaktiv (jede Minute) und reaktiv (nach der Erkennung von Bashlite). Diese Bewertungsszenarien haben gezeigt, dass das vorgeschlagene Framework und die Techniken (kooperativ und reaktiv) erhebliche Vorteile gegenüber einem unkooperativen und reaktiven Ansatz und einem kooperativen, aber proaktiven Ansatz haben. Neben der Halbierung der Gesamtinfektionszeit im System wurde auch die Verfügbarkeit der Maschinen, definiert durch ausgehende Paketverluste und ausgehende und eingehende Telnet-Verbindungen, insgesamt deutlich verbessert. Darüber hinaus war die CPU- und RAM-Auslastung des Frameworks und der ausgeführten Techniken minimal. Obwohl der kooperative und reaktive Ansatz bei weitem die besten Ergebnisse lieferte, hat jeder MTD-Ansatz seine Vorteile, und weitere Forschung ist erforderlich, um diesen vielversprechenden Abwehrmechanismus entsprechend zu nutzen.


\selectlanguage{english} 





