\chapter{Conclusion}
This thesis proposed and implemented a cooperative Moving Target Defence (MTD) framework for IoT devices along with two MTD techniques. IoT devices are inherently susceptible to malware for several reasons, including poor maintenance and neglect of security \cite{article:DDoSinIoT}. This insecurity, combined with the immense number of existing IoT devices \cite{website:statistaIoT}, makes these devices a popular target for malware. Infected IoT devices can be used for various malicious behaviours, the most common use case being botnets to perform DDoS attacks, for example. One way to defend IoT devices against malware is MTD, that attempts to change the attack surface of a system to defend it against attackers \cite{navas:2021MTDWhere}. \cite{article:vonderAssen} showed that it is possible to disconnect a Bashlite bot from the Bashlite server by changing the private IP address of the client. The solution proposed in this thesis builds on this by adding a cooperative component, as this provides a significant advantage to the defence mechanism.

In order to find an effective defence, a suitable malware had to be found. The choice fell on Bashlite, a fairly well-known IoT malware. Even though Bashlite was available on code repositories, some adaptations had to be made to the code to provide, among other things, the spreading functionality. With Bashlite working, the cooperative MTD mechanism could be developed. As a test bed, three virtual machines running Raspberry OS were used, one as a server machine and the other two to mimic the IoT devices. Of these two mimicking machines, one was manually infected (VM1) and the other was the susceptible machine (VM2) that was eventually infected through the spreading of Bashlite. 

The implemented solution consists of two different MTD techniques, which are the IP address change and the Telnet service port change, as IoT malware often infects devices via Telnet. The currently infected device performs an IP address change to disconnect itself from the Bashlite server, making communication with the server impossible. All other devices in a given IP address range should change their Telnet service port for a specified time to hide from Bashlite.

After the implementation, the solution was evaluated using several metrics, including the overall infection time of Bashlite on the two virtual machines, the network interruption caused by the execution of the MTD, and the RAM and CPU usage. These three metrics were measured in three different environments for comparison. The first environment was an non-cooperative and reactive MTD approach. This environment executed the MTD techniques 10 seconds after Bashlite was detected, but did not include a cooperative component. This means that Bashlite first had to be detected on the machine in order to disrupt its connection to the Bashlite server. The second environment was a cooperative and reactive MTD approach. Again, the MTD techniques were executed 10 seconds after Bashlite was detected, but here with the cooperative component. This was the solution developed in this thesis. The third and final environment was a cooperative but proactive environment. This was used for comparison and to show the trade-off between a proactive and a reactive MTD approach. This proactive approach ran the MTD techniques after a specified time interval (1 minute), regardless of whether Bashlite was on the system or not. 

The results showed that in the non-cooperative and reactive approach, the susceptible machine was infected on every run of Bashlite. This meant that both machines had to change their IP address to disconnect from the Bashlite server, resulting in each machine being infected for approximately 29 seconds. The cooperative and reactive approach completely prevented the spreading of Bashlite to the susceptible machine by changing the Telnet service port of the susceptible machine. Therefore, only the manually infected machine was infected for 29 seconds and the susceptible machine was not infected at all. This meant that the overall duration of infection activity on the two VMs was halved. Although not tested, this would be even more significant with more susceptible machines, as any infection other than the first could be prevented. Finally, the cooperative but proactive approach also prevented Bashlite from spreading to the vulnerable machine with the chosen configuration. However, it also showed a large variance in Bashlite duration on the first machine, as randomness partly determines after what time the MTD techniques are initiated. This resulted in an average Bashlite duration of 34 seconds on VM1 and 0 seconds on VM2. The results of this approach are strongly influenced by the configuration chosen, i.e. the interval at which the MTD techniques are initiated and other variables. It is also possible, for example, that the spreading of Bashlite cannot be prevented at all if the interval between the execution of the MTD techniques is too long.

Besides the overall infection time, the downtime of the machines is an important metric. The results clearly showed that a cooperative and reactive approach gave better results in terms of downtime than the uncooperative and reactive approach, as long as Telnet is not required on the susceptible machine. The cooperative and reactive approach also caused much less downtime (about 50\%) than the proactive and cooperative approach, which was to be expected. The final metric was the CPU and RAM usage of the MTD techniques. It turned out that both machines used on average a maximum of 0.43\% more CPU with the MTD mechanism deployed than without it and that the RAM usage of 10000 MB on each machine was negligible. 

Although the cooperative and reactive approach performed best in the evaluation, the other approaches should also be kept in mind. This is due to the different variables that should be considered when deciding on the most suitable MTD approach. Each approach has different advantages and disadvantages. For example, a short-interval proactive approach may be the best solution if the overall infection time of the system is to be minimized, and high downtime of connections to and from the machines is not an issue. In addition, it may be that the malware in question is not (yet) detectable, in which case the proactive approach would be required anyway.
However, for most systems, the cooperative and reactive approach will provide the best overall solution.

With the threat of IoT malware increasing both in number \cite{website:KasperskyAMalwareStory} and with new highly dangerous variants such as P2P malware \cite{website:trendMicroUncleanable}, it is essential to have the best possible defence options. One of these options is the Moving Target Defense which has proven to be a promising defence strategy for IoT devices, especially when it includes a cooperative component, as shown in this thesis. To be prepared for future threats and especially against P2P malware, further research on how cooperative Moving Target Defense can defend against IoT malware is essential.


\section{Future Research} \label{section:futureResearch}

As the topic of MTD is large and complex, there are several possibilities for further research. This thesis has shown that the cooperative and reactive MTD approach offers several advantages over other approaches. Examples of these advantages are a comparatively lower duration of infection activity on the system and a comparatively lower interruption of availability. A prerequisite for a reactive MTD approach to work is the ability to detect potential malware on the system. Therefore, research on fast and reliable malware detection mechanisms is essential. One possible means for that are machine learning algorithms~\cite{article:vonderAssen}.

Another research option is to further develop the presented solution. Possible improvements are to make the solution faster, for example by optimizing the \textit{nmap} scan, or to adapt the solution for other operating systems. A significant research would be to investigate if and to what extent the current solution works against malware such as Mirai or P2P botnets. Theoretically, moving the Telnet service port should also protect against these malwares, but this should be tested in practice.       

Another interesting area of research would be to further investigate the impact of the LAN setting. This has only been touched on slightly in this thesis, but the LAN setting could provide additional options for the malware that could have a negative impact on an MTD defence mechanism. For example, one of these potential options are port scanners that might be able to find the moved Telnet port.

Finally, the presented solution uses a classic client-server architecture. This was due to the simplicity provided, which allowed the basic concept of cooperative MTD to be tested more quickly. For a more sophisticated implementation, other architectures should be considered. For example, one option is 
a P2P architecture, which would provide the corresponding advantages, such as e.g. the removal of the single point of failure.



