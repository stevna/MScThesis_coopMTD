\chapter{Introduction}



\section{Motivation}
In today's world, IoT devices are already ubiquitous and can be found in several major end-use industries such as healthcare, manufacturing or banking~\cite{website:fortuneIoT}. In 2021, there were 11.3 billion IoT devices worldwide and this number is expected to triple by 2030~\cite{website:statistaIoT}. These devices facilitate the integration of the physical world into a computer-based system, providing various benefits such as improved efficiency and economic advantages~\cite{report:cern}. However, alongside the many benefits, there are also major security concerns, as the devices are popular targets for malware for a number of reasons. These include poor maintenance by manufacturers and weak/repetitive device passwords~\cite{article:DDoSinIoT,website:IBMIoT}. Infected IoT devices are ideal as bots for botnets, which can then launch e.g. distributed denial of service attacks. \cite{website:KasperskyAMalwareStory}~detected 105 million attacks on its honeypots in the first half of 2019, which is a significant increase from the 12 million attacks registered in the first half of 2018. The security problems, the rapidly growing number of devices worldwide and the increasing number of attacks make a strong defence option indispensable.  

One possible defence solution is Moving Target Defence (MTD). This is a cybersecurity paradigm that was first proposed in 2009~\cite{navas:2021MTDWhere}. Its aim is to constantly change the attack surface of a target to diminish the probability of a successful attack~\cite{navas:2021MTDWhere}. Examples include dynamic network techniques, which change network properties such as a device's IP address, or dynamic data techniques, which aim to change data representations~\cite{article:okhraviFindingFocus}. 

IoT malware such as Mirai or Bashlite include a spreading functionality~\cite{article:evOfBashlite}. This allows the malware to quickly spread to new devices, which in turn increases the power of the botnet at hand. This thesis aims to take this spreading functionality into account and create a cooperative MTD framework that mitigates/prevents the malware in a more effective way than a non-cooperative MTD framework is capable of.

\section{Description of Work}
This thesis proposes, implements and evaluates a cooperative MTD framework with two different MTD techniques. The framework is based on~\cite{article:Cedeno}, but includes a cooperative component to mitigate the selected malware and prevent its spreading. The selected malware is Bashlite, a well-known command and control malware, which was modified and completed for this thesis. After analyzing the weaknesses of Bashlite for possible levers where MTD techniques can be applied, two MTD techniques are selected and implemented. The first is to change the IP address of the infected device, which permanently disrupts communication with the Bashlite control server. The second technique is to temporarily change the Telnet service port of other susceptible machines in the network. As many IoT malware infect other devices via the Telnet port, temporarily moving the Telnet service port of the susceptible devices prevents the devices from being found in the first place. After a specified number of seconds, when the IP address change has finished securing the infected device, the Telnet service port changes back to port 23.

The framework and techniques are evaluated in three different scenarios using three different metrics. These metrics include the overall infection time of the system, the interruption of the availability of the machines and the CPU and RAM usage needed by the MTD framework and its techniques. The base case for the evaluation is a non-cooperative and reactive MTD framework that only uses the IP address change to clear the devices after Bashlite is found on them. Reactive means that there is a detection mechanism for the malware and the MTD techniques start as soon as the malware is found on the system. So in the first scenario, each machine has to be infected before it can do anything against Bashlite. The second scenario is cooperative and reactive. As soon as Bashlite is found on a machine on the network, the infected machine initiates the IP address change and all other machines in a given IP range change their Telnet service port to a different port. This is the solution presented in this thesis. The third scenario is cooperative, but proactive. Proactive means that the MTD techniques are run at a specified interval, in this case 60 seconds. Following the evaluation, its results are presented and discussed in-depth. 


\section{Thesis Outline}
After this introductory chapter, the thesis continues with the necessary background information (Chapter 2) for the rest of this thesis. Various information is presented there, including a general overview of MTD and why IoT devices are susceptible to malware. This chapter is followed by the related research chapter (Chapter 3), which first presents the state of research of MTD in IoT, then introduces some MTD IoT frameworks, and finally introduces some MTD IoT mechanisms/techniques. Chapter 4 is the implementation chapter, which deals with the selection and adaptation of Bashlite and possible levers where MTD techniques could be applied to mitigate a Bashlite infection. Additionally, the final implementation is presented in detail in the implementation chapter. This final implementation is then evaluated in Chapter 5, which first introduces the methodology of the evaluation and then presents the results. The last two chapters of the thesis are the discussion and the conclusion. Additionally, some limitations are presented in the discussion and possible future research in the conclusion. 
