\chapter{Discussion} \label{chapter:discussion}
This chapter presents the discussion of the results shown in Section \ref{section:results}. Additionally, some of the limitations of the thesis are outlined.


\section{Interpretation of Results}
The results of the evaluation turned out as expected. The cooperative and reactive approach provided the best defence against Bashlite, as the overall infection activity was the shortest. It also made sense that the variance of the Bashlite durations was so low in the reactive approaches, because there is no stochastic process involved in the defence mechanisms. In contrast, the proactive environment involves randomness, resulting in different overall infection times for each run. Thus, as long as the MTD techniques execution is reactive and the hardware does not reach its limit, which is unlikely, the mitigation process will have a similar mitigation time for each run. 

This mitigation time was around 29 seconds in the evaluation, which was surprisingly high. 10 seconds are needed for detection~\cite{article:vonderAssen}, the remaining 19 seconds are needed for the mitigation process. A large part of these 19 seconds was caused by the \textit{nmap} scan on the \textit{MTD Deployer Server}. Minimising this scan time was beyond the scope of this thesis, but it would be achievable~\cite{website:nmapTiming} and should be implemented in a more sophisticated solution. Another convenient way to reduce the \textit{nmap} search time would be to change the range of possible IP addresses in the \textit{MTD Deployer Server} configuration file. The evaluation was done with a range of 90 IP addresses, whether this is a reasonable number depends on the system in which the MTD would be deployed.

The detection time is another crucial variable. In this thesis, a detection time of around 10 seconds was used, as it was shown that Bashlite could be detected in this time~\cite{article:vonderAssen}. If this time were longer (e.g. with another malware), it is possible that a reactive approach would not be able to prevent the malware from spreading. This could force a user to use a proactive defence mechanism instead of a reactive one. The proactive approach used in the evaluation was successful in preventing Bashlite from spreading to the susceptible machine with the variables selected. This is by no means certain. Several variables are decisive whether the spreading of Bashlite can be prevented, the most important ones are: the time interval after which Bashlite is executed randomly, the execution interval of the MTD mechanism, and the time after which the Telnet service port is moved back to port 23. 

If the system protection was the only criterion, it would be best to run the MTD techniques at a very short interval. For example, an MTD mechanism (e.g. IP address change) executed every second would result in an extremely short overall infection activity. While this is desirable, it is not feasible as the connections to and from the device would always be interrupted, defeating the purpose of the device (e.g. a sensor collecting data). Thus, there is usually a trade-off between system availability and security. 

This trade-off between availability and security was also reflected in the results. In terms of outgoing packet losses, the non-cooperative and reactive approach had the highest share of packet loss out of the total number of packets sent. This is evident as both machines had to change their IP addresses, which is the cause of the packet loss. 

The share of failed packet losses can also be approximated mathematically. In this non-cooperative and reactive environment, a run took 120 seconds on each machine, resulting in a total of 240 seconds in the system. Restarting the Ethernet adapter to force the machine to use the new IP address takes about 5 seconds per machine, making a total of 10 seconds. 10 seconds is about 4.2\% of 240, which is the approximate interruption percentage. This also explains why the packet loss share of all packets sent in the cooperative and reactive environment was exactly half of the packet loss in the non-cooperative and reactive environment. Again, the total time was 240 seconds, but this time only one machine changed its IP address. 5 seconds out of 240 seconds is 2.1\%. The same calculations can be done for the cooperative but proactive environment. Here only one machine had to change its IP address, but it did so every 60 seconds due to the proactive approach. As there are two machines again, this gives a total of 120 seconds in the system. 5 seconds is about 4.2\% of 120 seconds. 

So far, the cooperative and reactive approach performed best. This approach had the lowest total infection time and the lowest share of outgoing packet losses. This was different when looking at the share of failed incoming Telnet connections. Here the non-cooperative and reactive approach performed best. This is evident as this approach did not initiate port changes, which are the main cause of failed incoming Telnet connections. Although not clearly visible in the figures, the share of failed incoming Telnet connections for this non-cooperative and reactive environment is identical to the share of its packet loss (4.2\%). This is due to the reboot of the Ethernet adapter, as a working IP address is a prerequisite for successful Telnet connections. 

The cooperative and reactive environment had a much higher interruption share of incoming Telnet connections due to the moved Telnet service port. It is important to note that this share was heavily influenced by the defined time to switch back to port 23, which can be specified in the \textit{MTD Deployer Server} configuration file. The 30 seconds selected for the evaluation were chosen because Bashlite would definitely be rendered harmless on the infected device and the port could therefore be moved back after 30 seconds. Again, the share of failed incoming Telnet connections was twice as high in the cooperative but proactive environment as in the cooperative and reactive environment. This is because the former ran the MTD every 60 seconds and the latter every 120 seconds. 



As mentioned earlier, there is a trade-off between security and availability and all the deployment strategies had their advantages. However, the only advantage that the non-cooperative reactive system had over the cooperative reactive system is the lower rate of failed incoming Telnet connections. Although it is unlikely that the availability of a device's Telnet port is more important than security and the outgoing packet loss rate, a solution in this case would be to use an uncooperative defence.

The cooperative and reactive approach also showed better results in the evaluation than the cooperative but proactive approach. However, this case is more complex as it depends on various factors such as the detection time of the malware or again the trade-off between security and availability. The proactive approach could theoretically provide better security results than the reactive approach, but this would also increase the interruption of the services. In a real system, the requirements of the underlying system would have to be compared with the advantages and disadvantages of the MTD solution. One way of doing this is to calculate the interruption time and decide which approach is more suitable for the system at hand. 



%As the results showed, the cooperative and reactive approach can definitely be a powerful tool against IoT malware. It significantly reduced the overall infection activity on a system and also had some advantages in terms of interruption of machine availability compared to an non-cooperative and reactive approach. The evaluation was done with only one susceptible machine, the difference would be even more significant with more than one susceptible machine in the network. In general, it is important to note that each deployment strategy had its advantages. Therefore, it is essential that the underlying system, as well as the possible threats to the system, are analysed prior to any potential real-world deployment to determine which approach provides the best overall solution with the fewest downsides.
In general, the cooperative and reactive approach can be recommended for most systems, as this will only run the MTD techniques if Bashlite has been found on the system. This avoids unnecessary interruption of the system that would occur with the proactive approach. However, it is impossible to make a final statement, as the choice also depends on the malware. If a really aggressive malware is trying to infect and possibly destroy the devices, a proactive approach may still be more reasonable. Nevertheless, the possibilities offered by a cooperative defence mechanism are very promising. 

Even in case a proactive approach is the required solution, there is no need to worry about the resources of the device, as it was found that the MTD framework and executed techniques used only a minimal amount of hardware resources. This is evident from the fact that it does not take many hardware resources to execute the \textit{sed} command to replace something in a file and then restart either the Ethernet adapter or the \textit{inetutils-inetd} service. The peaks seen in the CPU usage of VM1 with the MTD deployed are almost certainly caused by the restart of VM1's Ethernet adapter. It is not clear what caused the peaks of VM2. The reason may be the restart of the \textit{inetutils-inetd} service, but as they were not as periodic as VM1's peaks, it is difficult to determine.
However, these peaks were extremely short and therefore not a problem even for resource-constrained devices. In terms of RAM usage, the results were similar. The MTD framework and the executed techniques used only about 10,000 bytes of RAM. This is negligible compared to the 3.7 GB the machines had at their disposal, as it is less than 0.0003\%.

The results of this evaluation could also be usefully combined with two of the frameworks presented in Section \ref{section:MTDFramework}. The first was the framework that helps to answer the design questions (What, How, When) for an MTD mechanism. The What and the How are already determined by the implemented MTD mechanism, as well as the When for the reactive case, since this is determined by the detection time. However, the When for the proactive approach could definitely be determined with the help of this framework. The second framework was the IANVS framework, which aims to help implement MTD techniques in distributed systems. Whether this is needed or not depends on the system. As long as the connections of the IoT devices are outgoing and the target is static, such a mechanism is not needed because the IoT device can still connect to the target regardless of what IP address the IoT device currently has. However, if a device needs to connect to the IoT device, such a mechanism needs to be implemented. 




\section{Limitations} \label{section:limitations}
There are some important limitations in several aspects of this thesis. The first is the operating system of the virtual machines. As described, the initial setup was with the Ubuntu operating system, which had to be changed to the Raspberry OS in order to run Bashlite. This change caused the original implementation of the IP address change to fail. Although the Python parts of the solution should work independently of the OS, it is possible that the Bash commands executed in the Python code may need to be adapted on a different OS. This is especially important as there are many different potential operating systems for constrained devices, as shown in Section \ref{sec:hardSoftware}.

Another limitation is the chosen malware. In this thesis, Bashlite was used as the malware to work with. However, there exist more sophisticated malware that may be able to evade the defence mechanisms presented. Although this could only be evaluated through the code on GitHub, a malware such as Mirai has implemented the functionality to reconnect to its server when the client does not receive a response from the server. This makes the IP address change more of a short-term obstacle than a long-term solution. Additionally, it is not entirely clear how the IP address change would perform against a P2P malware such as the HEH malware described in Section \ref{subsection:P2PIoTBotnets}. At the very least, the Telnet port change technique is, in theory, a well working defence against malware such as HEH or Mirai. However, this also needs to be thoroughly tested in practice. Additionally, it is worth noting that malware can also target a device's SSH port. This thesis has only focused on the Telnet port.    

As already described in more detail in the previous Chapter (\ref{chapter:discussion}), the results of the evaluation were strongly influenced by the chosen variables and the configuration file (e.g. the proactive execution rhythm or the IP address range to be scanned by \textit{nmap}). Although this is normal in such an evaluation, it is important to emphasise this when talking about the limitations. Even though the variables selected for this thesis were carefully chosen, it is not possible to draw an absolute conclusion from the result. It is clear that in most cases it makes sense to use a cooperative and reactive approach where possible, but there may be environments/systems where the  cooperative and proactive approach is more suitable. Various variables determine how well an MTD approach performs, and different MTD approaches may have different advantages and disadvantages. It is therefore important to tailor the chosen MTD approach to the requirements of the system that is to be protected by the MTD solution.


